 \begin{frame}
  \frametitle{L'algorithme C4.5}
En bref : 
\begin{itemize}
 \item Algorithme dû à Ross Quinlan.
 \item Extension de son précédent algorithme ID3.
 \item Méthode générant un arbre de décision.
\end{itemize}
Pour construire un noeud de l'arbre : 
\begin{itemize}
 \item Choix d'un attribut qui sépare le mieux l'ensemble d'apprentissage.
 \item Critère : Entropie relative de chaque attribut.
 \item L'attribut qui a l'entropie relative la plus forte est choisi pour séparer l'ensemble d'instances.
\end{itemize}
Points forts de C4.5 :
\begin{itemize}
 \item Peut traiter des attributs discrets comme continus.
 \item Peut traiter des valeurs manquantes.
 \item Arbres plus petits grâce à un élagage.
\end{itemize}


 \end{frame}
